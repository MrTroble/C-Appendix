\documentclass{article}

\usepackage[utf8]{inputenc}
\usepackage{hyperref}
\usepackage{blindtext}
\usepackage{enumerate}
\usepackage{minted}
\usepackage{tabu}
\begin{document}
\title{C-Appendix}
\author{Benjamin Steffen, Nico Fröhlich}
\maketitle
\clearpage

\tableofcontents

\newpage

\section{Die Programmiersprache C}
C wurde 1972-1973 von Dennis Ritchie an den Bell Labs entwickelt. C ist eine systemnahe Sprache, da sie sehr nahe an der Hardware liegt. Sie gehört zu den am weitesten verbreiteten Sprachen und findet Anwendung in Betriebssystemen,
Nutzeranwendungen,  Integrierten Systemen uvm..\\
Dieses Dokument bescheibt eine Abwandlung von C wie sie der SDCC nutzt. Dies ist allerdings nicht der offizielle Standard, dieser wird in \\\textit{The C Programming Language (Dennis Ritchie, Brian Kernigham)} beschrieben.

\subsection{Datentypen}
C kennt folgende primitive Datentypen:
\begin{center}
\begin{tabular}{| l | c | r |}
	\hline
	\textbf{Datentyp} & \textbf{Speichergr{\"o}{\ss}e} & \textbf{Wertebereich} \\ \hline
    bool & 1 Byte & 0 oder 1 \\
	char & 1 Byte & -128 bis 127 \\
	short & 2 Bytes & -32,768 bis 32,767 \\
	int & 2 Bytes &  -32,768 bis 32,767 \\
	long & 4 Bytes & -2,147,483,648 bis 2,147,483,647 \\
	long long & 8 Bytes & -9223372036854775808 bis 9223372036854775807 \\
	float & 4 Bytes & 1.2E-38 bis 3.4E+38 (6 Dezimalstellen) \\
	double & 8 Bytes & 2.3E-308 bis 1.7E+308 (15 Dezimalstellen) \\
	\hline
\end{tabular}
\end{center}
Bis auf float und double haben all diese Datentypen auch eine \textit{unsigned} Variante. Das bedeutet, dass dann keine negative Werte, aber doppelt so viele positive Werte möglich sind. Desweiteren gibt es noch den Datentyp \textit{void}, welcher das Fehlen eines Datentyps darstellt.
Normalerweise gibt es auch keinen booleschen Datentyp, dann stellt 0 den Wert \textit{false} dar, während alles andere \textit{true} ist

\subsection{Pointer}
Ein Merkmal und mächtiges Werkzeug von C sind \textit{Pointer}. Pointer sind Variablen die keine alleinstehenden Werte beinhalten, sondern auf die Speicheradresse einer anderen Variable zeigen.
Die Adresse einer variable kann mithilfe des \textit{\&-Operators ermittelt werden}. Der \textit{*-Operator} hat zwei Aufgaben. Zum einen zeigt er an, dass eine Variable ein Pointer ist, zum anderen nutzt man ihn um den Wert eines Pointers auszulesen (Dereferenzieren).
Hat man also beispielsweise einen Integerwert in einem \textbf{int wert = 22} gespeichert, so kann man den Ort, an dem sich dieser Wert befindet in einem Pointer \textbf{int *ptr = \&wert} festhalten. Hierbei muss beachtet werden, dass weiterhin der Datentyp angegeben wird.
Grund dafür ist die unterschiedliche Größe der Datentypen. Intern wird zwischen ihnen nicht unterschieden, deshalb muss dem Compiler gesagt werden, wie viele Bytes gelesen oder geschrieben werden sollen. Hier sind Pointer zwischen einem und vier Bytes groß.

\subsection{Strings}
In C werden Zeichenketten (Strings) wörtlich genommen, dort gibt es nämlich keine Stringklasse wie in Java. Hier wird lediglich ein Array an Zeichen verwendet bzw. ein \textit{char*}, wobei die Zeichenkette bei dem ASCII code 0 bzw \textit{'\textbackslash 0'} endet.
Angegeben wird diese wie in Java im code mit "\newline Beispiel: \textit{char* string = "Hello, World!";}.

\subsection{Compiler und Linker}
Anders als Java ist C nicht interpretiert, sondern wird direkt in Maschinencode umgewandelt. Dieser Vorgang ist generell in zwei Schritte unterteilt. Zuerst wird der C Code Kompiliert, also in Maschinencode übersetzt. Dabei entstehen sogenannte \textit{Object Files}. Der Linker bringt diese Dateien dann in
einer Ausführbaren Datei zusammen und stellt Referenzen zwischen Funktionen und Bibliotheken her.
\subsection{Header}
Neben den Sourcedateien (.c) gibt es auch \textit{Headerdateien} (.h). Diese Dateien werden genutzt um Funktions- und Strukturprototypen zu deklarieren und diese durch eine Bibliothek oder eine andere Sourcedatei zugänglich zu machen.
Headerdateien werden über \textit{\#include {\textless}header.h{\textgreater}} eingebunden. \\Beim Kompilationsprozess wird die hier angegebene Datei einfach an die Stelle des include Befehls kopiert.

\section{Syntax}
Im folgenden wird kurz aufgezeigt, wie das Deklarieren und Verwenden von Variablen und Kontrollstrukturen aussieht. Beispiel:
\newpage
\begin{minted}{c}
#include <stdio.h> //Einbinden eines Headers

int main(int argc, char** argv) {	//Definition des Programmeinstiegpunktes
    int zahl = 12345;             //Deklaration verschiedenster Variablen
    float kommazahl = 123.45;      	
    char charakter = 'a';
    char *text = "Beispiel Text\0";  // \0 Markiert das Ende eines Characterstrings,
                                     // ist aber je nach implementation nicht immer nötig
    for(i=0; i<10; i++) { // Da wir einen Compiler haben der sich nicht an Standards 
    //hält darf man keinen Datentyp bei for schleifen angeben
        printf("Hello World! %d", i);	//10 malige ausgabe eines Textes,
    }                                 //mit integer Wert in einem Formatstring
    
	
    while(zahl>0)	//Blöcke {} sind nicht notwendig,
        zahl--;   //wenn unter dem Schleifenkopf nur eine Anweisung steht
		
	    return 0; 
	    //Die Funktion ended mit "return 0;" um erfolgreiche Abarbeitung zu melden
}
\end{minted}

\subsection{Schleifen und Verzweigungen}
In diesem Abschnitt werden Schleifen und Verzweigungen in C erklärt. Diese sind weitgehend wie in Java. Bei diesem Speziellen Compiler gibt es allerdings ein paar Eigenarten. [Da sich die Leute nicht an Standards halten können]
\subsubsection*{While}
\begin{minted}{c}
while(a == b) {
continue;// Muss mindestens ein Statement haben
}
\end{minted}
Infinite loop
\begin{minted}{c}
while(1) { // Kein true -> es gibt kein true (1 für true / 0 für false)
continue;// Muss mindestens ein Statement haben
}
\end{minted}
Do-While
\begin{minted}{c}
do {
continue;// Muss mindestens ein Statement haben
} while (a != b)
\end{minted}
\subsubsection*{For}
\begin{minted}{c}
for(i = 0; i < size; i++) { // Kein Datentyp
continue;// Muss mindestens ein Statement haben
}
\end{minted}
\subsubsection*{If}
\begin{minted}{c}
if (a == b) {
continue;// Muss mindestens ein Statement haben
}
\end{minted}

\subsection{Literals}
\begin{center}
\begin{tabular}{| l | c | r |}
	\hline
	\textbf{Literal} & \textbf{Nutzung} &  \textbf{Beispiel} \\ \hline
    0b & Binärschreibweise & 0b0001 für 1\\
    0x & Hexadezimalschreibweise & 0x000F für 16\\
	0* & Oktalschreibweise & 012 für 10\\
	\hline
\end{tabular}
\end{center}

\subsection{Operator}
Die logischen sowie die bitweisen Operatoren sind in C genau wie in Java.
\begin{center}
\begin{tabular}{| l | c | r |}
	\hline
	\textbf{Operator} & \textbf{Beschreibung} &  \textbf{Beispiel} \\ \hline
    \&\& & Logisches und & (a \&\& b)\\
    ! & Logisches nicht & !a\\
	\begin{math}
    \vert\vert \end{math} & Logisches oder & (a \begin{math}
    \vert\vert \end{math} b)\\
    +, += & Addition (und setzen)& a + b, a += b\\
    -, -= & Subtraktion (und setzen)& a - b, a -= b\\
    \begin{math} *, *=\end{math} & Multiplikation (und setzen)& \begin{math} a * b, a *= b\end{math}\\
    /, /= & Division (und setzen)& a / b, a /= b\\
    \%, \%= & Modulo (und setzen)& a \% b, a \%= b\\
    \& & Bitweise AND & 0b011 \& 0b010\\
\begin{math}| \end{math}& Bitweise inklusive ODER &\begin{math}0b011 | 0b010\end{math}\\
\textasciicircum & Bitweise exklusiv ODER & 0b011 \textasciicircum 0b010 \\
\textasciitilde & Bitweise Negation & \textasciitilde0\\
\textless\textless & Bitweise links verschieben & 1 \textless\textless 3\\
\textgreater\textgreater & Bitweise links verschieben & 4 \textgreater\textgreater 2\\
	\hline
\end{tabular}
\end{center}

\subsection{C8051F340.h}
Siehe: https://github.com/MrTroble/ADC-C8051F340/blob/master/C8051F340.h
\newline\newline
Der C8051F340 Header ist ein von Cygnal/SiLabs bereit gestellter Header der bereits alle ASM Register mit ihren jeweiligen Adressen enthält. Dabei können die jeweiligen Register wie normale variablen mit bit Operatoren angesteuert werden. Beispiel:
\begin{minted}{c}
#include <C8051F340.h>

void main() {
    ADC0CN |= 0x80; // Enable ADC0 subsystem
    AMX0N = 0x1F; // Set negative mux to GND -> use single end mode
    // use AMX0P default config -> P1.0
    while (1) { // Infinite loop
        ADC0CN |= 0x10; // or 16 to trigger AD0BUSY
        while (!(ADC0CN & 0x20 /* or 32*/)) {continue;} // wait for the AD0INT to become 1
        // go ahead after thread block
        P3 = ADC0L; // Put low bits in P3
        P0_0 = ADC0H & 1; // Put first bit of high bits into P0.0
        P0_1 = ADC0H & 2; // Put second bit of high bits into P0.1
    }
}
\end{minted}
\end{document}
